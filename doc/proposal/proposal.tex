\documentclass[12pt, a4paper, twoside]{report}
\usepackage{hyperref}
\usepackage{amsmath}
\usepackage{multirow}
\usepackage{subcaption}
\usepackage{listings}
\usepackage{caption}
\usepackage{pdflscape}
% font size could be 10pt (default), 11pt or 12 pt
% paper size coulde be letterpaper (default), legalpaper, executivepaper,
% a4paper, a5paper or b5paper
% side coulde be oneside (default) or twoside 
% columns coulde be onecolumn (default) or twocolumn
% graphics coulde be final (default) or draft 
%
% titlepage coulde be notitlepage (default) or titlepage which 
% makes an extra page for title 
% 
% paper alignment coulde be portrait (default) or landscape 
%
% equations coulde be 
%   default number of the equation on the rigth and equation centered 
%   leqno number on the left and equation centered 
%   fleqn number on the rigth and  equation on the left side
%   
\lstset{basicstyle=\footnotesize, frame=single}

\title{Proposal for an Agda UHC Backend}
\author{Philipp Hausmann \\
    4003373 \\
    }

\date{\today} 
\begin{document}


\maketitle

\tableofcontents

\chapter{General Idea}
The dependently typed language Agda is lacking a decent compiler at the time of writing. While there
exist three backends for Agda, of which one is maintained, all three have their shortcomings.
The aim of this proposal is to create a new backend for Agda, using the UHC Core language
as intermediate language to produce efficient executables. Apart from just being able to compile
Agda code, it shall also be researched how a FFI interface between Agda and Haskell can be constructed,
what restrictions apply to such a FFI binding and how it may interfere with optimizations.

\chapter{Existing Backends}
The current version of Agda has 3 backends. The MAlonzo backend targets Haskell as intermediate
language, which is then compiled using GHC. The JS backend, as it's name says, targets javascript
and is intended for compiling web applications. Lastly, the Epic backend targets the Epic language
and uses the Epic compiler to produce an executable.
We shall now look at these three backends a bit more in depth.
\section{MAlonzo}
MAlonzo is the only officially maintained backend of Agda and is able to compile the whole standard
library. This backend contains no optimizations, apart from special handling of some Agda builtin
datatypes like Nat. Instead, the MAlonzo compiler relies on GHC in that regard, which depending
on the Agda program works to differing degress. Especially in situations where the type information
present in Agda would be required to perform optimizations, the limitations of the current implementation
become apparent. (Nat example)

While this problems can be overcome, it also needs to be mentioned that the target language Haskell
is not a perfect fit for compiling Agda. As Haskell doesn't support dependent types, the produced
code contains a lot of "coerce" statements to cheat the haskell type system. Notwithstanding any
bugs in the Agda backend, this should be safe but shows that there is a mismatch between the
two languages.

\chapter{Prototype Compiler}
A prototype backend for Agda has been created, and it has been shown that it is feasible to compile
Agda using UHC Core. This prototype lacks any optimizations and contains only a rudimentary FFI
interface, which is unidirectional (Agda to Haskell). It is also restricted to compiling just one
Agda module, and support for coinduction is missing too.

\chapter{Proposal}
\section{Agda Module support}
\subsection{Parameterised Modules}
Parametrized Agda modules can be mapped to normal plain modules, where every function get's the
modules parameter as additional arguments. (see \url{http://wiki.portal.chalmers.se/agda/pmwiki.php?n=ReferenceManual.Modules#param})

\section{Haskell FFI}
\subsection{Type Erasure}

TODO it seems that the indexes of data types are already get erased?
For functions, this is not the case on the other hand.
\begin{landscape}
\begin{figure}
\begin{subfigure}[b]{0.5\linewidth}
\lstinputlisting{code/TypeErasureAgda.agda}
\begin{lstlisting}
data TypeErasureAgda.agda_List
  =   TypeErasureAgda.List.agda_nil = {0,0}
    , TypeErasureAgda.List.agda_cons = {1,2}
    ;
let rec
  { TypeErasureAgda.agda_List
      = \h0  ->
          #Tag {Rec}
  ; TypeErasureAgda.List.agda_nil
      = #Tag {TypeErasureAgda.agda_List, nil, 0}
  ; TypeErasureAgda.List.agda_cons
      = \x  xs  ->
          (#Tag {TypeErasureAgda.agda_List, cons, 1})
            (x)
            (xs)
  } in
...
\end{lstlisting}
\end{subfigure}
\begin{subfigure}[b]{0.5\linewidth}
\lstinputlisting[firstline=3]{code/TypeErasureHs.hs}
\begin{lstlisting}
data TypeErasureHs.List'
  =   TypeErasureHs.Cons = {0,2}
    , TypeErasureHs.Nil = {1,0}
    ;
let
  { TypeErasureHs.Nil
      = #Tag {TypeErasureHs.List', TypeErasureHs.Nil, 1}
  } in
let
  { TypeErasureHs.Cons
      = \x  xs  ->
          (#Tag {TypeErasureHs.List', TypeErasureHs.Cons, 0})
            (x)
            (xs)
  } in
...
\end{lstlisting}
\end{subfigure}
\end{figure}
\end{landscape}

\subsection{Parameterised Modules}
The Haskell Module system does have no concept of parameterised modules, therefore another representation is required. In the end,
all functions will have to get passed the module parameters explicitly. This could be done either by just prepending the module
argument list to all function definitions, or by wrapping the module parameters in a datatype and passing that around.

How should datatypes be handled?

\subsection{Class System}
\paragraph{Calling Agda from Haskell}
Simplistic approach -> pass dictionaries by hand
Sophisticated -> Generate class and instance definitions from Agda

\paragraph{Calling Haskell from Agda}
Generate Agda Record definitions and instances using the Agda Instance arguments feature.
This should be sound.

\section{Optimizations}
Inductivies families, bradley
type erasure
relevance analysis
irrelevance
forcing (same as bradley paper)?
natish data types (also, when is plus a plus?)
integer-ish data types
TODO: smash, primitivise, caseopts

\section{Contracts (for FFI)}

let f be the Contract, of type "a -> b -> .... -> Y -> Dec X"

test' : a -> c -> b -> Y
test' ... = FFI CALL

test : a -> c -> b -> assertT (f, X)
test a c b = let y = assertV (f a b) (test' a c b)

\section{UHC Type Checker}
we need to typecheck calls from UHC to Agda. I guess we should somehow generate a HI file for Agda files?

\section{Agda Type Checker}
is there enough info inside the HI file for the Agda type checker? I doubt it...

\section{Unified Compilation}
let SUPERUC be the unified compilier, then a set of Haskell and/or Agda files
could be compiled by passing them to SUPERUC. It would be possible to import
Agda modules from HS and the other way around and compile them in one go.
We probably don't want to support cyclic dependencies, and the FFI restrictions still apply.

Probably interferes with the UHC/Agda module chasing algorithm.

\section{Cabal Integration}
long term, it would be nice to be able to compile Agda programs using Cabal and
mix it with plain Haskell packages.

\section{UHC Changes}
For the proposal laid out so far, certain changes in UHC itself appear
unavoidable. A preliminary list of this changes is as follows:

\subsection{Index Datatypes by Aspects}
TODO just an idea...
Binds inside expressions can be indexed by aspects, which does
not apply to data types. However, there may be multiple definitions
of the same data type caused by worker/wrapper style transformations/optimizations.
While this definitions could be encoded using a naming convention, a more explicit
approach may be preferrable.

\subsection{Module support for Core}
At the moment, UHC can only compile exactly one core file to an executable.
While this core file may import modules from other packages, it is not possible
to import other Core files or build a package from Core files.

\subsection{UHC Library (EH99) on hackage}
UHC uses a traditional "./configure \&\& make" build process. If the parts used by the
Agda backend could be split of into a normal cabal package residing on hackage,
making this library a hard dependency of Agda might be possible.
It would also allow the cabal dependency solver to figure out a compatible
set of packages for both UHC and Agda, which may goes wrong in the current setup.
(Cabal right now needs some manual guidance to pick packages compatible with both UHC and Agda)

\subsection{Cabal support for UHC}
The cabal support for installing packages into the UHC database is broken. If this would work, it
would marginally simplfy the installation of the uhc-agda-base package.

\begin{thebibliography}{1}

\bibitem{FS91}
  Peter W. Frey, David J. Slate.
  \emph{Letter Recognition Using Holland-Style Adaptive Classifier}.
  Kluwer Academic Publishers,
  Machine Learning, 6, 161-182(1991).
  Tom Dietterich, Boston,
  1991.

\end{thebibliography}

\end{document}
